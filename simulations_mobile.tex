\documentclass[a5paper]{scrartcl}
\usepackage[solutionfiles]{optional}

% Als je daarentegen de antwoorden onder de opdracht wil hebben, quote
% dan bovenstaande uit, en gebruik de regel hieronder
%\usepackage[nosolutionfiles]{optional}

\opt{nosolutionfiles}{\usepackage[nosolutionfiles]{answers}}
\opt{solutionfiles}{\usepackage{answers}}

\usepackage[english]{babel}

\usepackage{mathtools,amsthm,amssymb}
\usepackage{verbatim}
\usepackage[colorlinks=true]{hyperref}

\usepackage[T1]{fontenc}
\usepackage{fouriernc}
\usepackage{fontawesome} % for the linux hint symbol
\newcommand{\hintsymbol}{\marginpar{\center{\faLinux}}}


% I don't know why, pythontext should appear somewhere above some
% other packages. I don't know which ones, but at this spot it works.
\usepackage{pythontex}

\usepackage{fancyhdr}
\pagestyle{fancy}
\fancyhead{} % clear all header fields
\fancyhead[LO, LE]{\rightmark}
\fancyfoot{} % clear all footer fields
\fancyfoot[C]{\thepage}
\setlength{\headheight}{14pt} 

\Newassociation{solution}{Solution}{ans}
%\Newassociation{sol}{Solution}{ans}
\Newassociation{hint}{Hint}{hint}
\renewcommand{\Hintlabel}[1]{\textbf{h.#1}}
\renewcommand{\Solutionlabel}[1]{\textbf{s.#1}}

\theoremstyle{definition} % font type for theorems and related environments
\newtheorem{exercise}{Exercise}[section]

\usepackage{etoolbox}% necessary for the command below
\AtBeginEnvironment{hint}{\hintsymbol} 

\newcommand{\E}[1]{\,\mathsf{E}\left[#1\right]}


\usepackage[top=5mm, left=5mm, right=5mm, bottom=2cm]{geometry}

\input{main_text.tex}

%%% Local Variables:
%%% mode: latex
%%% TeX-master: t
%%% End:
